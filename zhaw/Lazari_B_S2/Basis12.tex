\question
Lösen Sie die folgende Aufgabe manuell auf einem Blatt Papier und scannen Sie dieses in die Datei \textit{Name\_Klasse\_S2\_Aufg1.pdf}:
\begin{parts}
\part
Geben Sie für die reelle Dezimalzahl $x_0 = 118\,559.999$ die Maschinenzahl $\tilde{x_0}$ in normalisierter Gleitpunktdarstellung zur Basis 12 (Duodezimalsystem) mit Mantisselänge $n=7$ und hinreichend grossem Exponenten an (verwenden Sie dazu die Ziffern $0, 1, \dots, 9$ sowie $A  \triangleq 10, B \triangleq 11$). Wie gross ist der absolute und relative Fehler, der bei der Abbildung auf die Maschinenzahl entsteht?
\begin{solutionordottedlines}[2cm]
i.) Vorkommaanteil
\begin{center}
\begin{tabular}{rrrr}
\toprule
\multicolumn{1}{c}{\textbf{Wert}} & \multicolumn{1}{c}{\textbf{Divident}} & \multicolumn{1}{c}{\textbf{Ganzzahl}} & \multicolumn{1}{c}{\textbf{Rest}}\\\midrule
118\,559 & 12 & 9879 & $11 = B$\\\hline
9\,879 & 12 & 823 & 3 \\\hline
823 & 12 & 68 & 7 \\\hline
68 & 12 & 5 & 8 \\\hline
5 & 12 & 0 & 5 \\\bottomrule
\end{tabular}
\end{center}
Vorkommaanteil = $5873B$

ii.) Nachkommaanteil
\begin{center}
\begin{tabular}{rrrr}
\multicolumn{1}{c}{\textbf{Wert}} & \multicolumn{1}{c}{\textbf{Multiplikator}} & \multicolumn{1}{c}{\textbf{Resultat}} & \multicolumn{1}{c}{\textbf{Ganzzahl}}\\\midrule
0.999 & 12 & 11.998 & $11 = B$\\\hline
0.988 & 12 & 11.856 & $11 = B$\\\hline
0.856 & 12 & 10.272 & $10 = A$\\\hline
0.272 & 12 & 3.264  & 3 \\\hline
\dots & 12 & \dots  & \dots \\\bottomrule
\end{tabular}
\end{center}
Nachkommaanteil = $BBA3$

iii.) Zusammen: $x_0 = 118\,559.999_{10} \approx 5873B.BBA3_{12}$

iv.) Normalisiert: $x_0 \approx 0.5873BBBA3_{12} \cdot 12^5$

v.) Mit 7-stelliger Mantisse:  $\tilde{x} = 0.5873BBB_{12} \cdot 12^5$

vi.) Rückkonvertiert: 
\begin{multline*}
5 \cdot 12^4 + 8 \cdot 12^3 + 7 \cdot 12^2 + 3 \cdot 12^1 + 11 \cdot 12^0 + \\
11 \cdot 12^{-1} + 11 \cdot 12^{-2} = 118\,560.83\overline{3}_{10}	
\end{multline*}
vii.) Absoluter Fehler: $|\tilde{x} - x_0| = |118\,560.83\overline{3}_{10} - 118\,560.999_{10}| = 0.1656\overline{6}$
viii.) Relativer Fehler: $\frac{|\tilde{x} - x_0|}{x_0} = 1.3973_{10} \cdot 10^{-6}$
\end{solutionordottedlines}
%%% Next subquestion

\pagebreak
\part
Berechnen Sie nun den Funktionswert $f(x) = x^3 - 1.6665 \cdot 10^{15}$ sowohl für $x_0$ als auch für $\tilde{x_0}$. Wie gross ist der relative Fehler der Funktionswerte?
\begin{solutionordottedlines}[2cm]
\begin{align*}
	f(x) = f(118\,560.999)& = 118\,559.999^3 - 1.6665 \cdot 10^{15} \approx 1.5776 \cdot 10^{12}\\
	f(\tilde{x}) = f(118\,560.83\overline{3}) & = 118\,560.83\overline{3}^3 - 1.6665 \cdot 10^{15} \approx 1.5707 \cdot 10^{12}\\
	\frac{|f(\tilde{x}) - f(x)|}{f(x)} = 0.004428251
\end{align*}
\end{solutionordottedlines}
%%% Next subquestion

\part
Berechnen Sie die Konditionszahl und vergleichen Sie diese mit dem Verhältnis der relativen Fehler aus a) und b). Gab die Konditionszahl in diesem Beispiel eine realistische Abschätzung der Fehlerfortpflanzung?
\begin{solutionordottedlines}[2cm]
\begin{align*}
	K &:= \frac{|f'(\tilde{x})| \cdot |\tilde{x}|}{|f(\tilde{x})|}\\
	f'(\tilde{x})& = 3 \cdot 118\,560.83\overline{3}^2 = 4\,217\,013\,602.0833\\
	K& = \frac{4\,217\,013\,602.0833 \cdot 118\,560.83\overline{3}}{1.5707 \cdot 10^{12}} = 3\,183.209084
\end{align*}
Die Konditionszahl ist sehr hoch. Wir haben mit 3 Nachkommastellen gearbeitet und der Fehler wirkt sich bereits auf die dritte Nachkommastelle aus. Somit halt ich die Abschätzung für realistisch.
\end{solutionordottedlines}
%%% Next subquestion

\end{parts}