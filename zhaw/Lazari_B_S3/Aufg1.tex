\documentclass[12pt, a4paper, answers]{exam} % Lösungen
%\documentclass[12pt, addpoints, a4paper]{exam} % Ohne Lösungen
\newcommand{\myAuthor}{Constantin Lazari}
\newcommand{\myTitle}{Übungen Numerik}
\newcommand{\mySubject}{Numerik I (2013)}
\newcommand{\myNumber}{3}

\usepackage[utf8]{inputenc}
\usepackage[pdftex]{graphicx} 
\usepackage{microtype}
\usepackage[pdfborder={0 0 0}, plainpages=false, pdfpagelabels]{hyperref} 
\usepackage[ngerman]{babel}
\usepackage[babel]{csquotes}
\usepackage{tabularx} 

\usepackage{tikz}
\usetikzlibrary{arrows,automata}
\usepackage{amsmath,amssymb,amsthm}

\usepackage{lmodern} %Type1-Schriftart fuer nicht-englische Texte
\hyphenation{eine einer eines} % Trennung von eine, einer, eines vermeiden
\usepackage{microtype}

\usepackage{color}
\usepackage{stmaryrd}
\usepackage{booktabs}


%% Listings %%%%%%%%%%%%%%%%%%%%%%%%%%%%%%%%%%%%%%%%%%%%%%%%%
%\usepackage{verbatim}
\usepackage{listings}
\lstloadlanguages{[LaTeX]TeX}
{\lstset{%
  basicstyle=\footnotesize\ttfamily,
  commentstyle=\slshape\color{green!50!black},
  keywordstyle=\bfseries\color{blue!50!black},
  identifierstyle=\color{blue},
  stringstyle=\color{orange},
  escapechar=\#,
  emphstyle=\color{red}}
}
{
  \lstset{%
    basicstyle=\ttfamily,
    keywordstyle=\bfseries,
    commentstyle=\itshape,
    escapechar=\#,
    emphstyle=\bfseries\color{red}
  }
}

\hypersetup{
	pdfauthor   = {\myAuthor},
	pdftitle    = {\myTitle},
	pdfsubject  = {\mySubject},
	pdfkeywords = {},
	pdfcreator  = {Kile},
	pdfproducer = {pdflatex},
	colorlinks 	= false
} 

\setlength{\parindent}{0em}
\setlength{\parskip}{0.75em}

%% Exam Settings
\pagestyle{headandfoot}
%\firstpageheader{Benutzer/innen im Umgang mit Informatikmitteln instruieren}{}{Lernkontrolle 1}
\firstpageheader{\mySubject}{}{Übung \myNumber}
\firstpageheadrule

%\runningheader{Benutzer/innen im Umgang mit Informatikmitteln instruieren}{}{Lernkontrolle 1}
\runningheader{\mySubject}{}{
\ifprintanswers
  Lösung Übung \myNumber
\else
  Übung \myNumber
\fi
}
\runningheadrule

\firstpagefooter{}{Seite \thepage\ von \numpages}{}
\firstpagefootrule

\runningfooter{}{Seite \thepage\ von \numpages}{}
\runningfootrule

\pointsinrightmargin
\pointpoints{Punkt}{Punkte}
\bonuspointpoints{Bonuspunkt}{Bonuspunkte}
\renewcommand{\solutiontitle}{\noindent\textbf{Lösung:}\par\noindent}

\CorrectChoiceEmphasis{\bfseries}
\renewcommand\choicelabel{$\boxempty$}

\begin{document}
	\begin{tabularx}{\textwidth}{Xr}
	\myAuthor & \today\\
	\end{tabularx}
  %% Lernkontrolle einbinden
	\begin{questions}
\question
Ist das Potenzieren ($f(x) = x^n, n \in \mathbb{N}$) bzw. das Wurzelziehen ($f(x) = \sqrt[n]{x}, n \in \mathbb{N}$)
einer reelen Zahl $x$ gut oder schlecht konditioniert? Begründen Sie! Was hat das für Auswirkungen auf die Auswertung von Polynomen für grosse $n$?
\begin{parts}
\part
Potenzieren
\begin{solutionordottedlines}[2cm]
\begin{align*}
K &= \frac{f(x)}{f'(x)} \cdot x\\
  &= \frac{x^n}{n \cdot x^{n-1}} \cdot x = \frac{x^n}{n \cdot x^n \cdot x^{-1}} \cdot x = \frac{x^2}{n}
\end{align*}
Je grösser $n$, desto kleiner wird $K$. Somit ist es gut konditioniert, und grössere Polynome lassen sich besser auswerten als kleine. 
\end{solutionordottedlines}
%%% Next subquestion

\part
Wurzelziehen
\begin{solutionordottedlines}[2cm]
\begin{align*}
K &= \frac{f(x)}{f'(x)} \cdot x\\
  &= \frac{x^{1/n}}{1/n \cdot x^{1/n -1}} \cdot x = \frac{x^{1/2}}{1/n \cdot x^{1/n} \cdot x^{-1}} \cdot x = n \cdot x^2
\end{align*}
Je grösser $n$, desto grösser wird $K$. Somit ist es schlecht konditioniert, und grössere Wurzel lassen sich nur ungenau berechnen.
\end{solutionordottedlines}
%%% Next subquestion

\end{parts}
	\end{questions}
\end{document}