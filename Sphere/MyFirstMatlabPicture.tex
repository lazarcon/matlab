% Source: howtoTeX.com
% Feel free to distribute, but leave a reference to howtoTeX.com
% Date: April 2012
\RequirePackage[l2tabu, orthodox]{nag}
\documentclass
[
a4paper,	% alle weiteren Papierformat einstellbar
%landscape,	% Grundsätzlich im Querformat
11pt,		% Schriftgrösse (12pt, 11pt (Standard))
BCOR1cm,	% Bindekorrektur, bspw. 1 cm
DIV=14,		% Satzspiegel neu berechnen
%oneside,	% einseitiges Layout
%twocolumn,	% zweispaltiger Satz
%openany,	% Kapitel können auch auf linken Seiten beginnen
%halfparskip*,	% Absatzformatierung s. scrguide 3.1
%headsepline,	% Trennline zum Seitenkopf
%footsepline,	% Trennline zum Seitenfuss
%notitlepage,	% Keine eigene Titelseite
%chapterprefix,	% Vor Kapitelüberschrift wird "Kapitel Nummer" gesetzt
%appendixprefix,	% Anhang wird "Anhang" vor die Überschrift gesetzt 
%normalheadings,% Überschriften etwas kleiner (smallheadings)
%idxtotoc,	% Index im Inhaltsverzeichnis
%listof=totoc,	% Abb.- und Tabellen Verzeichnis im Inhaltsverzeichnis
bibliography=totoc,	% Literaturverzeichnis im Inhalt
%leqno,		% Nummerierung von Gleichungen links
%fleqn,		% Ausgabe von Gleichungen linksbündig
%draft		% Überlangen Zeilen in Ausgabe gekennzeichnet
%chapterprefix=false	% Keine "Kapitel"' am Anfang eines Kapitels
ngerman
]{scrartcl}

%%%%%%%%%%%%%%%%%%%%%%%%%%%%%%%%%%%%%%%%%%%%%%%%%%%%%%%%%%%%%%%%%%%%
\def \myAuthor {Constantin Lazari}
\def \myContext {MATLAB}
\def \mySubject {Erste Schritte}
\def \myTitle {Eine kugelige Sache}
\def \myDate {\today}
%%%%%%%%%%%%%%%%%%%%%%%%%%%%%%%%%%%%%%%%%%%%%%%%%%%%%%%%%%%%%%%%%%%%

%% Deutsche Spezialitäten
\usepackage[T1]{fontenc}
\usepackage[utf8]{inputenc}
\usepackage{babel}
\usepackage[babel,german=swiss]{csquotes} % Oder german=quotes
\hyphenation{eine einer eines} % Trennung von eine, einer, eines vermeiden

\usepackage{geometry}
\geometry{verbose,a4paper,
	tmargin=20mm,
	bmargin=20mm,
	lmargin=20mm,
	rmargin=25mm,
	headsep=5mm,
	footskip=5mm
}
\setlength{\skip\footins}{1em}
\renewcommand{\baselinestretch}{1.50}\normalsize
\renewcommand*{\dictumwidth}{.66\textwidth}

\usepackage[singlespacing]{setspace} %singlespacing, onehalfspacing, doublespacing

%% Schrift und Schriftverbesserung
\usepackage{lmodern} %Type1-Schriftart fuer nicht-englische Texte
\usepackage{textcomp}
\usepackage{microtype}

%% Wissenschaftliches
\usepackage{amssymb,amsmath,amsthm,mathtools}
\usepackage[mode=text,per-mode=symbol,exponent-product = \cdot]{siunitx}
\sisetup{locale=DE}

%% Informatik: Listings
\usepackage{listings}
\lstloadlanguages{Java}
\lstset{%
basicstyle	=\footnotesize\ttfamily,
keywordstyle=\bfseries,
commentstyle=\itshape,
escapechar	=\#,
emphstyle	=\bfseries\underline,
tabsize		=4,
breaklines	=true,
}
%% Umlaute richtig behandeln
\lstset{
  literate=	{ö}{{\"o}}1
           	{ä}{{\"a}}1
           	{ü}{{\"u}}1
		   			{Ö}{{\"O}}1
		  	 		{Ä}{{\"A}}1
		   			{Ü}{{\"U}}1
}

%% Referenzen
\usepackage{nameref}

%% Graphik
\usepackage{pgfplots}
%\usepackage{tikz}
%\usetikzlibrary{matrix,arrows,decorations.pathmorphing}
\usepackage{graphicx} %[pdftex] entfernt, wenn pgf-plots verwendet wird 

%% Tabellen
\usepackage{booktabs}

%% Misc Styling
\usepackage{tocloft}
\renewcommand{\cftsecleader}{\cftdotfill{\cftdotsep}}

\titlehead{\myContext}
\subject{\mySubject}
\title{\myTitle}
\author{\myAuthor}
\date{\myDate}

\usepackage{bookmark, hyperref} %bookmark,  
\usepackage[final]{pdfpages}
\pdfoutput=1
\hypersetup{
	pdfauthor   = {\myAuthor},
	pdftitle    = {\myTitle},
	pdfsubject  = {\mySubject},
	pdfkeywords = {},
	pdfcreator  = {Kile},		% Texnic Center oder Kile z.B.
	pdfproducer = {pdflatex},	
	pdfborder	= 0 0 0,
	colorlinks  = false		% Links nicht farbig hervorheben (sieht Scheisse aus).
} 

%% Kopf- und Fusszeilen %%%%%%%%%%%%%%%%%%%%%%%%%%%%%%%%%%%%%%%%%%%%%%%
\usepackage{totpages}
\usepackage{fancyhdr}
\fancyhf{}

%\renewcommand\chaptermark[1]{\markboth{#1}{}} 	% Disable chapter numbers
\renewcommand\sectionmark[1]{\markboth{#1}{}} 	% Disable section numbers
\renewcommand*{\partheadmidvskip}{~}

% Defaultstyle - Head
\fancyhead[L]{\nouppercase{\myTitle}}
\fancyhead[C]{}
\fancyhead[R]{\nouppercase{\leftmark}}
\renewcommand{\headrulewidth}{0.5pt} % Toprule
% Defaultstyle - Foot
\fancyfoot[L]{\myDate}
\fancyfoot[C]{\myAuthor} 
\fancyfoot[R]{\thepage/\pageref{TotPages}}
\renewcommand{\footrulewidth}{0.5pt} % Bottomrule

% Style 'Plain'
\fancypagestyle{plain}{%
  \fancyhf{}
  % Plainstyle - Head
  \fancyhead[L]{}
  \fancyhead[C]{}
  \fancyhead[R]{}
  \renewcommand{\headrulewidth}{0pt} % Hide toprule
  % Plainstyle - Foot
  \fancyfoot[L]{\myDate}   
  \fancyfoot[C]{\myAuthor} 
  \fancyfoot[R]{\thepage/\pageref{TotPages}}
  %\renewcommand{\footrulewidth}{0.5pt} % Hide bottomrule
}
\pagestyle{fancy}

\setlength{\parindent}{0em}
\setlength{\parskip}{0.75em}
\setlength{\skip\footins}{1em}

%% Schusterjungen und Hurenkinder vermeiden
\clubpenalty = 10000
\widowpenalty = 10000
\displaywidowpenalty = 10000

%% Eigene Macros
%%%%%%%%%%%%%%%%%%%%%%%%%%%%%%%%%%%%%%%%%%%%%%%%%%%%%%%%%%%%%%%%%%%%%%%%%%%%%

% Ausführlicher Verweis (siehe Abschnitt x: Titel, Seite x)
\newcommand{\see}[2]{(siehe #1 \ref{#2}: \nameref{#2}, Seite \pageref{#2})}

% Zentriertes fett geschriebener Spaltenkopf
\newcommand{\colhead}[2]{\multicolumn{#1}{c}{\textbf{#2}}}

% Datum im Iso-Format
\newcommand{\dateIso}[3]{#3-#2-#1}

% Datum im Din-Format
\newcommand{\dateDin}[3]{#1.\,#2.~#3}

% Richtige Abkürzung für zum Beispiel
\def \zB {z.\,B. }

% Richtige Abkürzung für das heisst
\def \dH {d.\,h. } 

% Richtige Abkürzung für unter anderem
\def \ua {u.\,a. }

%%%%%%%%%%%%%%%%%%%%%%%%%%%%%%%%%%%%%%%%%%%%%%%%%%%%%%%%%%%%%%%%%%%%%%%%%%%%%%%%
% Document
%%%%%%%%%%%%%%%%%%%%%%%%%%%%%%%%%%%%%%%%%%%%%%%%%%%%%%%%%%%%%%%%%%%%%%%%%%%%%%%%

\begin{document}

\maketitle

\begin{abstract}
	Eine Kugel, erstellt mit Matlab
\end{abstract}

\tableofcontents

%\listoftables

\listoffigures

%\lstlistoflistings
%\pagebreak

\section{Matlab-Grafik}
\begin{figure}[h!t]
	\centering 
	\newlength\figureheight 
	\newlength\figurewidth 
	\setlength\figureheight{6cm} 
	\setlength\figurewidth{6cm} 
	\caption{Matlab-Plot einer Kugel}
	% This file was created by matlab2tikz v0.4.1.
% Copyright (c) 2008--2013, Nico Schlömer <nico.schloemer@gmail.com>
% All rights reserved.
% 
% The latest updates can be retrieved from
%   http://www.mathworks.com/matlabcentral/fileexchange/22022-matlab2tikz
% where you can also make suggestions and rate matlab2tikz.
% 
% 
% 
\begin{tikzpicture}

\begin{axis}[%
width=\figurewidth,
height=\figureheight,
view={-37.5}{30},
scale only axis,
xmin=-1,
xmax=1,
xmajorgrids,
ymin=-1,
ymax=1,
ymajorgrids,
zmin=-1,
zmax=1,
zmajorgrids,
axis x line*=bottom,
axis y line*=left,
axis z line*=left
]

\addplot3[%
surf,
colormap/jet,
shader=faceted,
draw=black,
z buffer=sort,
mesh/rows=21]
table[row sep=crcr,header=false] {
0 0 -1\\
-0.156434465040231 0 -0.987688340595138\\
-0.309016994374947 0 -0.951056516295154\\
-0.453990499739547 0 -0.891006524188368\\
-0.587785252292473 0 -0.809016994374947\\
-0.707106781186548 0 -0.707106781186547\\
-0.809016994374947 0 -0.587785252292473\\
-0.891006524188368 0 -0.453990499739547\\
-0.951056516295154 0 -0.309016994374947\\
-0.987688340595138 0 -0.156434465040231\\
-1 0 0\\
-0.987688340595138 0 0.156434465040231\\
-0.951056516295154 0 0.309016994374947\\
-0.891006524188368 0 0.453990499739547\\
-0.809016994374947 0 0.587785252292473\\
-0.707106781186548 0 0.707106781186547\\
-0.587785252292473 0 0.809016994374947\\
-0.453990499739547 0 0.891006524188368\\
-0.309016994374947 0 0.951056516295154\\
-0.156434465040231 0 0.987688340595138\\
0 0 1\\
0 0 -1\\
-0.148778017349658 -0.048340908203385 -0.987688340595138\\
-0.293892626146237 -0.0954915028125263 -0.951056516295154\\
-0.431770623113389 -0.140290779704295 -0.891006524188368\\
-0.559016994374947 -0.18163563200134 -0.809016994374947\\
-0.672498511963957 -0.218508012224411 -0.707106781186547\\
-0.769420884293813 -0.25 -0.587785252292473\\
-0.847397560890843 -0.275336158073158 -0.453990499739547\\
-0.904508497187474 -0.293892626146237 -0.309016994374947\\
-0.939347432391753 -0.305212482389889 -0.156434465040231\\
-0.951056516295154 -0.309016994374948 0\\
-0.939347432391753 -0.305212482389889 0.156434465040231\\
-0.904508497187474 -0.293892626146237 0.309016994374947\\
-0.847397560890843 -0.275336158073158 0.453990499739547\\
-0.769420884293813 -0.25 0.587785252292473\\
-0.672498511963957 -0.218508012224411 0.707106781186547\\
-0.559016994374947 -0.18163563200134 0.809016994374947\\
-0.431770623113389 -0.140290779704295 0.891006524188368\\
-0.293892626146237 -0.0954915028125263 0.951056516295154\\
-0.148778017349658 -0.048340908203385 0.987688340595138\\
0 0 1\\
0 0 -1\\
-0.1265581407235 -0.0919498715009102 -0.987688340595138\\
-0.25 -0.18163563200134 -0.951056516295154\\
-0.367286029574069 -0.266848920427796 -0.891006524188368\\
-0.475528258147577 -0.345491502812526 -0.809016994374947\\
-0.572061402817684 -0.415626937777454 -0.707106781186547\\
-0.654508497187474 -0.475528258147577 -0.587785252292473\\
-0.720839420167342 -0.523720494614299 -0.453990499739547\\
-0.769420884293813 -0.559016994374948 -0.309016994374947\\
-0.799056652687458 -0.580548640463047 -0.156434465040231\\
-0.809016994374947 -0.587785252292473 0\\
-0.799056652687458 -0.580548640463047 0.156434465040231\\
-0.769420884293813 -0.559016994374948 0.309016994374947\\
-0.720839420167342 -0.523720494614299 0.453990499739547\\
-0.654508497187474 -0.475528258147577 0.587785252292473\\
-0.572061402817684 -0.415626937777454 0.707106781186547\\
-0.475528258147577 -0.345491502812526 0.809016994374947\\
-0.367286029574069 -0.266848920427796 0.891006524188368\\
-0.25 -0.18163563200134 0.951056516295154\\
-0.1265581407235 -0.0919498715009102 0.987688340595138\\
0 0 1\\
0 0 -1\\
-0.0919498715009102 -0.1265581407235 -0.987688340595138\\
-0.18163563200134 -0.25 -0.951056516295154\\
-0.266848920427795 -0.367286029574069 -0.891006524188368\\
-0.345491502812526 -0.475528258147577 -0.809016994374947\\
-0.415626937777453 -0.572061402817684 -0.707106781186547\\
-0.475528258147577 -0.654508497187474 -0.587785252292473\\
-0.523720494614299 -0.720839420167342 -0.453990499739547\\
-0.559016994374947 -0.769420884293813 -0.309016994374947\\
-0.580548640463047 -0.799056652687458 -0.156434465040231\\
-0.587785252292473 -0.809016994374947 0\\
-0.580548640463047 -0.799056652687458 0.156434465040231\\
-0.559016994374947 -0.769420884293813 0.309016994374947\\
-0.523720494614299 -0.720839420167342 0.453990499739547\\
-0.475528258147577 -0.654508497187474 0.587785252292473\\
-0.415626937777453 -0.572061402817684 0.707106781186547\\
-0.345491502812526 -0.475528258147577 0.809016994374947\\
-0.266848920427795 -0.367286029574069 0.891006524188368\\
-0.18163563200134 -0.25 0.951056516295154\\
-0.0919498715009102 -0.1265581407235 0.987688340595138\\
0 0 1\\
0 0 -1\\
-0.0483409082033849 -0.148778017349658 -0.987688340595138\\
-0.0954915028125263 -0.293892626146237 -0.951056516295154\\
-0.140290779704295 -0.431770623113389 -0.891006524188368\\
-0.18163563200134 -0.559016994374947 -0.809016994374947\\
-0.21850801222441 -0.672498511963957 -0.707106781186547\\
-0.25 -0.769420884293813 -0.587785252292473\\
-0.275336158073158 -0.847397560890843 -0.453990499739547\\
-0.293892626146236 -0.904508497187474 -0.309016994374947\\
-0.305212482389889 -0.939347432391753 -0.156434465040231\\
-0.309016994374947 -0.951056516295154 0\\
-0.305212482389889 -0.939347432391753 0.156434465040231\\
-0.293892626146236 -0.904508497187474 0.309016994374947\\
-0.275336158073158 -0.847397560890843 0.453990499739547\\
-0.25 -0.769420884293813 0.587785252292473\\
-0.21850801222441 -0.672498511963957 0.707106781186547\\
-0.18163563200134 -0.559016994374947 0.809016994374947\\
-0.140290779704295 -0.431770623113389 0.891006524188368\\
-0.0954915028125263 -0.293892626146237 0.951056516295154\\
-0.0483409082033849 -0.148778017349658 0.987688340595138\\
0 0 1\\
0 0 -1\\
9.57884834439237e-18 -0.156434465040231 -0.987688340595138\\
1.89218336521708e-17 -0.309016994374947 -0.951056516295154\\
2.77989006174672e-17 -0.453990499739547 -0.891006524188368\\
3.59914663902998e-17 -0.587785252292473 -0.809016994374947\\
4.32978028117747e-17 -0.707106781186548 -0.707106781186547\\
4.95380036308546e-17 -0.809016994374947 -0.587785252292473\\
5.45584143933347e-17 -0.891006524188368 -0.453990499739547\\
5.82354159244546e-17 -0.951056516295154 -0.309016994374947\\
6.04784682432498e-17 -0.987688340595138 -0.156434465040231\\
6.12323399573677e-17 -1 0\\
6.04784682432498e-17 -0.987688340595138 0.156434465040231\\
5.82354159244546e-17 -0.951056516295154 0.309016994374947\\
5.45584143933347e-17 -0.891006524188368 0.453990499739547\\
4.95380036308546e-17 -0.809016994374947 0.587785252292473\\
4.32978028117747e-17 -0.707106781186548 0.707106781186547\\
3.59914663902998e-17 -0.587785252292473 0.809016994374947\\
2.77989006174672e-17 -0.453990499739547 0.891006524188368\\
1.89218336521708e-17 -0.309016994374947 0.951056516295154\\
9.57884834439237e-18 -0.156434465040231 0.987688340595138\\
0 0 1\\
0 0 -1\\
0.048340908203385 -0.148778017349658 -0.987688340595138\\
0.0954915028125263 -0.293892626146237 -0.951056516295154\\
0.140290779704295 -0.431770623113389 -0.891006524188368\\
0.18163563200134 -0.559016994374947 -0.809016994374947\\
0.218508012224411 -0.672498511963957 -0.707106781186547\\
0.25 -0.769420884293813 -0.587785252292473\\
0.275336158073158 -0.847397560890843 -0.453990499739547\\
0.293892626146237 -0.904508497187474 -0.309016994374947\\
0.305212482389889 -0.939347432391753 -0.156434465040231\\
0.309016994374947 -0.951056516295154 0\\
0.305212482389889 -0.939347432391753 0.156434465040231\\
0.293892626146237 -0.904508497187474 0.309016994374947\\
0.275336158073158 -0.847397560890843 0.453990499739547\\
0.25 -0.769420884293813 0.587785252292473\\
0.218508012224411 -0.672498511963957 0.707106781186547\\
0.18163563200134 -0.559016994374947 0.809016994374947\\
0.140290779704295 -0.431770623113389 0.891006524188368\\
0.0954915028125263 -0.293892626146237 0.951056516295154\\
0.048340908203385 -0.148778017349658 0.987688340595138\\
0 0 1\\
0 0 -1\\
0.0919498715009102 -0.1265581407235 -0.987688340595138\\
0.18163563200134 -0.25 -0.951056516295154\\
0.266848920427796 -0.367286029574069 -0.891006524188368\\
0.345491502812526 -0.475528258147577 -0.809016994374947\\
0.415626937777453 -0.572061402817684 -0.707106781186547\\
0.475528258147577 -0.654508497187474 -0.587785252292473\\
0.523720494614299 -0.720839420167342 -0.453990499739547\\
0.559016994374947 -0.769420884293813 -0.309016994374947\\
0.580548640463047 -0.799056652687458 -0.156434465040231\\
0.587785252292473 -0.809016994374947 0\\
0.580548640463047 -0.799056652687458 0.156434465040231\\
0.559016994374947 -0.769420884293813 0.309016994374947\\
0.523720494614299 -0.720839420167342 0.453990499739547\\
0.475528258147577 -0.654508497187474 0.587785252292473\\
0.415626937777453 -0.572061402817684 0.707106781186547\\
0.345491502812526 -0.475528258147577 0.809016994374947\\
0.266848920427796 -0.367286029574069 0.891006524188368\\
0.18163563200134 -0.25 0.951056516295154\\
0.0919498715009102 -0.1265581407235 0.987688340595138\\
0 0 1\\
0 0 -1\\
0.1265581407235 -0.0919498715009102 -0.987688340595138\\
0.25 -0.18163563200134 -0.951056516295154\\
0.367286029574069 -0.266848920427796 -0.891006524188368\\
0.475528258147577 -0.345491502812526 -0.809016994374947\\
0.572061402817684 -0.415626937777453 -0.707106781186547\\
0.654508497187474 -0.475528258147577 -0.587785252292473\\
0.720839420167342 -0.523720494614299 -0.453990499739547\\
0.769420884293813 -0.559016994374947 -0.309016994374947\\
0.799056652687458 -0.580548640463047 -0.156434465040231\\
0.809016994374947 -0.587785252292473 0\\
0.799056652687458 -0.580548640463047 0.156434465040231\\
0.769420884293813 -0.559016994374947 0.309016994374947\\
0.720839420167342 -0.523720494614299 0.453990499739547\\
0.654508497187474 -0.475528258147577 0.587785252292473\\
0.572061402817684 -0.415626937777453 0.707106781186547\\
0.475528258147577 -0.345491502812526 0.809016994374947\\
0.367286029574069 -0.266848920427796 0.891006524188368\\
0.25 -0.18163563200134 0.951056516295154\\
0.1265581407235 -0.0919498715009102 0.987688340595138\\
0 0 1\\
0 0 -1\\
0.148778017349658 -0.0483409082033849 -0.987688340595138\\
0.293892626146237 -0.0954915028125263 -0.951056516295154\\
0.431770623113389 -0.140290779704295 -0.891006524188368\\
0.559016994374947 -0.18163563200134 -0.809016994374947\\
0.672498511963957 -0.218508012224411 -0.707106781186547\\
0.769420884293813 -0.25 -0.587785252292473\\
0.847397560890843 -0.275336158073158 -0.453990499739547\\
0.904508497187474 -0.293892626146237 -0.309016994374947\\
0.939347432391753 -0.305212482389889 -0.156434465040231\\
0.951056516295154 -0.309016994374947 0\\
0.939347432391753 -0.305212482389889 0.156434465040231\\
0.904508497187474 -0.293892626146237 0.309016994374947\\
0.847397560890843 -0.275336158073158 0.453990499739547\\
0.769420884293813 -0.25 0.587785252292473\\
0.672498511963957 -0.218508012224411 0.707106781186547\\
0.559016994374947 -0.18163563200134 0.809016994374947\\
0.431770623113389 -0.140290779704295 0.891006524188368\\
0.293892626146237 -0.0954915028125263 0.951056516295154\\
0.148778017349658 -0.0483409082033849 0.987688340595138\\
0 0 1\\
0 0 -1\\
0.156434465040231 0 -0.987688340595138\\
0.309016994374947 0 -0.951056516295154\\
0.453990499739547 0 -0.891006524188368\\
0.587785252292473 0 -0.809016994374947\\
0.707106781186548 0 -0.707106781186547\\
0.809016994374947 0 -0.587785252292473\\
0.891006524188368 0 -0.453990499739547\\
0.951056516295154 0 -0.309016994374947\\
0.987688340595138 0 -0.156434465040231\\
1 0 0\\
0.987688340595138 0 0.156434465040231\\
0.951056516295154 0 0.309016994374947\\
0.891006524188368 0 0.453990499739547\\
0.809016994374947 0 0.587785252292473\\
0.707106781186548 0 0.707106781186547\\
0.587785252292473 0 0.809016994374947\\
0.453990499739547 0 0.891006524188368\\
0.309016994374947 0 0.951056516295154\\
0.156434465040231 0 0.987688340595138\\
0 0 1\\
0 0 -1\\
0.148778017349658 0.0483409082033849 -0.987688340595138\\
0.293892626146237 0.0954915028125263 -0.951056516295154\\
0.431770623113389 0.140290779704295 -0.891006524188368\\
0.559016994374947 0.18163563200134 -0.809016994374947\\
0.672498511963957 0.218508012224411 -0.707106781186547\\
0.769420884293813 0.25 -0.587785252292473\\
0.847397560890843 0.275336158073158 -0.453990499739547\\
0.904508497187474 0.293892626146237 -0.309016994374947\\
0.939347432391753 0.305212482389889 -0.156434465040231\\
0.951056516295154 0.309016994374947 0\\
0.939347432391753 0.305212482389889 0.156434465040231\\
0.904508497187474 0.293892626146237 0.309016994374947\\
0.847397560890843 0.275336158073158 0.453990499739547\\
0.769420884293813 0.25 0.587785252292473\\
0.672498511963957 0.218508012224411 0.707106781186547\\
0.559016994374947 0.18163563200134 0.809016994374947\\
0.431770623113389 0.140290779704295 0.891006524188368\\
0.293892626146237 0.0954915028125263 0.951056516295154\\
0.148778017349658 0.0483409082033849 0.987688340595138\\
0 0 1\\
0 0 -1\\
0.1265581407235 0.0919498715009102 -0.987688340595138\\
0.25 0.18163563200134 -0.951056516295154\\
0.367286029574069 0.266848920427796 -0.891006524188368\\
0.475528258147577 0.345491502812526 -0.809016994374947\\
0.572061402817684 0.415626937777453 -0.707106781186547\\
0.654508497187474 0.475528258147577 -0.587785252292473\\
0.720839420167342 0.523720494614299 -0.453990499739547\\
0.769420884293813 0.559016994374947 -0.309016994374947\\
0.799056652687458 0.580548640463047 -0.156434465040231\\
0.809016994374947 0.587785252292473 0\\
0.799056652687458 0.580548640463047 0.156434465040231\\
0.769420884293813 0.559016994374947 0.309016994374947\\
0.720839420167342 0.523720494614299 0.453990499739547\\
0.654508497187474 0.475528258147577 0.587785252292473\\
0.572061402817684 0.415626937777453 0.707106781186547\\
0.475528258147577 0.345491502812526 0.809016994374947\\
0.367286029574069 0.266848920427796 0.891006524188368\\
0.25 0.18163563200134 0.951056516295154\\
0.1265581407235 0.0919498715009102 0.987688340595138\\
0 0 1\\
0 0 -1\\
0.0919498715009102 0.1265581407235 -0.987688340595138\\
0.18163563200134 0.25 -0.951056516295154\\
0.266848920427796 0.367286029574069 -0.891006524188368\\
0.345491502812526 0.475528258147577 -0.809016994374947\\
0.415626937777453 0.572061402817684 -0.707106781186547\\
0.475528258147577 0.654508497187474 -0.587785252292473\\
0.523720494614299 0.720839420167342 -0.453990499739547\\
0.559016994374947 0.769420884293813 -0.309016994374947\\
0.580548640463047 0.799056652687458 -0.156434465040231\\
0.587785252292473 0.809016994374947 0\\
0.580548640463047 0.799056652687458 0.156434465040231\\
0.559016994374947 0.769420884293813 0.309016994374947\\
0.523720494614299 0.720839420167342 0.453990499739547\\
0.475528258147577 0.654508497187474 0.587785252292473\\
0.415626937777453 0.572061402817684 0.707106781186547\\
0.345491502812526 0.475528258147577 0.809016994374947\\
0.266848920427796 0.367286029574069 0.891006524188368\\
0.18163563200134 0.25 0.951056516295154\\
0.0919498715009102 0.1265581407235 0.987688340595138\\
0 0 1\\
0 0 -1\\
0.048340908203385 0.148778017349658 -0.987688340595138\\
0.0954915028125263 0.293892626146237 -0.951056516295154\\
0.140290779704295 0.431770623113389 -0.891006524188368\\
0.18163563200134 0.559016994374947 -0.809016994374947\\
0.218508012224411 0.672498511963957 -0.707106781186547\\
0.25 0.769420884293813 -0.587785252292473\\
0.275336158073158 0.847397560890843 -0.453990499739547\\
0.293892626146237 0.904508497187474 -0.309016994374947\\
0.305212482389889 0.939347432391753 -0.156434465040231\\
0.309016994374947 0.951056516295154 0\\
0.305212482389889 0.939347432391753 0.156434465040231\\
0.293892626146237 0.904508497187474 0.309016994374947\\
0.275336158073158 0.847397560890843 0.453990499739547\\
0.25 0.769420884293813 0.587785252292473\\
0.218508012224411 0.672498511963957 0.707106781186547\\
0.18163563200134 0.559016994374947 0.809016994374947\\
0.140290779704295 0.431770623113389 0.891006524188368\\
0.0954915028125263 0.293892626146237 0.951056516295154\\
0.048340908203385 0.148778017349658 0.987688340595138\\
0 0 1\\
0 0 -1\\
9.57884834439237e-18 0.156434465040231 -0.987688340595138\\
1.89218336521708e-17 0.309016994374947 -0.951056516295154\\
2.77989006174672e-17 0.453990499739547 -0.891006524188368\\
3.59914663902998e-17 0.587785252292473 -0.809016994374947\\
4.32978028117747e-17 0.707106781186548 -0.707106781186547\\
4.95380036308546e-17 0.809016994374947 -0.587785252292473\\
5.45584143933347e-17 0.891006524188368 -0.453990499739547\\
5.82354159244546e-17 0.951056516295154 -0.309016994374947\\
6.04784682432498e-17 0.987688340595138 -0.156434465040231\\
6.12323399573677e-17 1 0\\
6.04784682432498e-17 0.987688340595138 0.156434465040231\\
5.82354159244546e-17 0.951056516295154 0.309016994374947\\
5.45584143933347e-17 0.891006524188368 0.453990499739547\\
4.95380036308546e-17 0.809016994374947 0.587785252292473\\
4.32978028117747e-17 0.707106781186548 0.707106781186547\\
3.59914663902998e-17 0.587785252292473 0.809016994374947\\
2.77989006174672e-17 0.453990499739547 0.891006524188368\\
1.89218336521708e-17 0.309016994374947 0.951056516295154\\
9.57884834439237e-18 0.156434465040231 0.987688340595138\\
0 0 1\\
0 0 -1\\
-0.0483409082033849 0.148778017349658 -0.987688340595138\\
-0.0954915028125263 0.293892626146237 -0.951056516295154\\
-0.140290779704295 0.431770623113389 -0.891006524188368\\
-0.18163563200134 0.559016994374947 -0.809016994374947\\
-0.21850801222441 0.672498511963957 -0.707106781186547\\
-0.25 0.769420884293813 -0.587785252292473\\
-0.275336158073158 0.847397560890843 -0.453990499739547\\
-0.293892626146236 0.904508497187474 -0.309016994374947\\
-0.305212482389889 0.939347432391753 -0.156434465040231\\
-0.309016994374947 0.951056516295154 0\\
-0.305212482389889 0.939347432391753 0.156434465040231\\
-0.293892626146236 0.904508497187474 0.309016994374947\\
-0.275336158073158 0.847397560890843 0.453990499739547\\
-0.25 0.769420884293813 0.587785252292473\\
-0.21850801222441 0.672498511963957 0.707106781186547\\
-0.18163563200134 0.559016994374947 0.809016994374947\\
-0.140290779704295 0.431770623113389 0.891006524188368\\
-0.0954915028125263 0.293892626146237 0.951056516295154\\
-0.0483409082033849 0.148778017349658 0.987688340595138\\
0 0 1\\
0 0 -1\\
-0.0919498715009102 0.1265581407235 -0.987688340595138\\
-0.18163563200134 0.25 -0.951056516295154\\
-0.266848920427795 0.367286029574069 -0.891006524188368\\
-0.345491502812526 0.475528258147577 -0.809016994374947\\
-0.415626937777453 0.572061402817684 -0.707106781186547\\
-0.475528258147577 0.654508497187474 -0.587785252292473\\
-0.523720494614299 0.720839420167342 -0.453990499739547\\
-0.559016994374947 0.769420884293813 -0.309016994374947\\
-0.580548640463047 0.799056652687458 -0.156434465040231\\
-0.587785252292473 0.809016994374947 0\\
-0.580548640463047 0.799056652687458 0.156434465040231\\
-0.559016994374947 0.769420884293813 0.309016994374947\\
-0.523720494614299 0.720839420167342 0.453990499739547\\
-0.475528258147577 0.654508497187474 0.587785252292473\\
-0.415626937777453 0.572061402817684 0.707106781186547\\
-0.345491502812526 0.475528258147577 0.809016994374947\\
-0.266848920427795 0.367286029574069 0.891006524188368\\
-0.18163563200134 0.25 0.951056516295154\\
-0.0919498715009102 0.1265581407235 0.987688340595138\\
0 0 1\\
0 0 -1\\
-0.1265581407235 0.0919498715009102 -0.987688340595138\\
-0.25 0.18163563200134 -0.951056516295154\\
-0.367286029574069 0.266848920427796 -0.891006524188368\\
-0.475528258147577 0.345491502812526 -0.809016994374947\\
-0.572061402817684 0.415626937777454 -0.707106781186547\\
-0.654508497187474 0.475528258147577 -0.587785252292473\\
-0.720839420167342 0.523720494614299 -0.453990499739547\\
-0.769420884293813 0.559016994374948 -0.309016994374947\\
-0.799056652687458 0.580548640463047 -0.156434465040231\\
-0.809016994374947 0.587785252292473 0\\
-0.799056652687458 0.580548640463047 0.156434465040231\\
-0.769420884293813 0.559016994374948 0.309016994374947\\
-0.720839420167342 0.523720494614299 0.453990499739547\\
-0.654508497187474 0.475528258147577 0.587785252292473\\
-0.572061402817684 0.415626937777454 0.707106781186547\\
-0.475528258147577 0.345491502812526 0.809016994374947\\
-0.367286029574069 0.266848920427796 0.891006524188368\\
-0.25 0.18163563200134 0.951056516295154\\
-0.1265581407235 0.0919498715009102 0.987688340595138\\
0 0 1\\
0 0 -1\\
-0.148778017349658 0.048340908203385 -0.987688340595138\\
-0.293892626146237 0.0954915028125263 -0.951056516295154\\
-0.431770623113389 0.140290779704295 -0.891006524188368\\
-0.559016994374947 0.18163563200134 -0.809016994374947\\
-0.672498511963957 0.218508012224411 -0.707106781186547\\
-0.769420884293813 0.25 -0.587785252292473\\
-0.847397560890843 0.275336158073158 -0.453990499739547\\
-0.904508497187474 0.293892626146237 -0.309016994374947\\
-0.939347432391753 0.305212482389889 -0.156434465040231\\
-0.951056516295154 0.309016994374948 0\\
-0.939347432391753 0.305212482389889 0.156434465040231\\
-0.904508497187474 0.293892626146237 0.309016994374947\\
-0.847397560890843 0.275336158073158 0.453990499739547\\
-0.769420884293813 0.25 0.587785252292473\\
-0.672498511963957 0.218508012224411 0.707106781186547\\
-0.559016994374947 0.18163563200134 0.809016994374947\\
-0.431770623113389 0.140290779704295 0.891006524188368\\
-0.293892626146237 0.0954915028125263 0.951056516295154\\
-0.148778017349658 0.048340908203385 0.987688340595138\\
0 0 1\\
0 0 -1\\
-0.156434465040231 0 -0.987688340595138\\
-0.309016994374947 0 -0.951056516295154\\
-0.453990499739547 0 -0.891006524188368\\
-0.587785252292473 0 -0.809016994374947\\
-0.707106781186548 0 -0.707106781186547\\
-0.809016994374947 0 -0.587785252292473\\
-0.891006524188368 0 -0.453990499739547\\
-0.951056516295154 0 -0.309016994374947\\
-0.987688340595138 0 -0.156434465040231\\
-1 0 0\\
-0.987688340595138 0 0.156434465040231\\
-0.951056516295154 0 0.309016994374947\\
-0.891006524188368 0 0.453990499739547\\
-0.809016994374947 0 0.587785252292473\\
-0.707106781186548 0 0.707106781186547\\
-0.587785252292473 0 0.809016994374947\\
-0.453990499739547 0 0.891006524188368\\
-0.309016994374947 0 0.951056516295154\\
-0.156434465040231 0 0.987688340595138\\
0 0 1\\
};
\end{axis}
\end{tikzpicture}% 
	\label{fig:sphere} 
	\end{figure}
\end{document}